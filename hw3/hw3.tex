\documentclass[12pt,letterpaper]{article}
\usepackage{fullpage}
\usepackage[top=2cm, bottom=4.5cm, left=2.5cm, right=2.5cm]{geometry}
\usepackage{amsmath,amsthm,amsfonts,amssymb,amscd}
\usepackage{lastpage}
\usepackage{enumerate}
\usepackage{fancyhdr}
\usepackage{mathrsfs}
\usepackage{xcolor}
\usepackage{graphicx}
\usepackage{hyperref}
\usepackage{sectsty}
\usepackage{enumitem}
\usepackage{siunitx}
\usepackage{stackengine}
\usepackage{subcaption}
\usepackage{scalerel}
\usepackage{pdfpages}


\sectionfont{\fontsize{14.4}{17}\selectfont}
\subsectionfont{\fontsize{10}{12}\selectfont}

\hypersetup{%
  colorlinks=true,
  linkcolor=blue,
  linkbordercolor={0 0 1}
}


\newcommand\ddfrac[2]{\frac{\displaystyle #1}{\displaystyle #2}}


\setlength{\parindent}{0.0in}
\setlength{\parskip}{0.05in}

% Edit these as appropriate
\newcommand\course{CMPE 670}
\newcommand\hwnumber{3}
\newcommand\NetIDa{Andrei Tumbar}

\pagestyle{fancyplain}
\headheight 35pt
\lhead{\NetIDa}
\chead{\textbf{\Large Homework \hwnumber}}
\rhead{\course \\ \today}
\lfoot{}
\cfoot{}
\rfoot{\small\thepage}
\headsep 1.5em

\renewcommand{\theenumi}{\Alph{enumi}}


% Long division commands
\newcommand\showdiv[1]{\overline{\smash{\hstretch{.5}{)}\mkern-3.2mu\hstretch{.5}{)}}#1}}
\let\ph\phantom

\begin{document}

\section*{Question 1}

\begin{enumerate}[label=\alph*)]
\item Packet 112 is sent by the client.
\item The server's IP is 216.75.194.220 and its port is 443 (https).
\item The next TCP sequence number will be 283. This is because the current
      sequence number is 79 and the message length is 204, $79+204=283$.
\item Packet 112 contains 3 SSL records.
\item Packet 112 contains an encrypted master secret: "Encrypted Handshake Message". 
\end{enumerate}

\section*{Question 2}
In mobile IP, the mobile device must first forward to a base station which is
then connected to the greater network. Mobile IP will also need to contend with
channel utilization and conflicts more than wired does.

\section*{Question 3}
\begin{enumerate}[label=\alph*)]
\item
\begin{align*}
p &= 5\\
q &= 11\\
n = pq &= \mathbf{55}\\
z = (p-1)(q-1) = 4 \cdot 10 &= \mathbf{40} 
\end{align*}

\item 3 is a good choice for $e$ because it shares no common factors with z (40).
\item To find $x\mod 40 = 1$, there are only a couple of options that are less that
160: $1, 41, 81, 121 $. The only number divisible by 3 is 81. Therefore $d$ must be
$27$.
\item
\begin{align*}
c &= m^e \mod n\\
c &= 8^3 \mod 55 = \mathbf{17}
\end{align*}
\end{enumerate}

\section*{Question 4}
\begin{enumerate}[label=\alph*)]
\item False
\item True
\item True
\item False
\end{enumerate}

\section*{Question 5}
\begin{enumerate}[label=\alph*)]
\item The reciever will still be able to retrieve the original message even
      if one of the aggregate pulses was corrupted. This is because the 8
      aggregate pulses are redundant and the average of the code will decode
      into the original signal.
\item Even if two of the eight pulses were corrupted, DS/CDMA would still be
      able to extract the original message.
\item The reciever will multiple each bit by the chipping sequence. Each of these
      These numbers are then averaged over the number of bits in the chipping
      sequence. Ideally, a "1" bit will result in the chipping sequence and a "-1"
      bit will result in an inverted chipping sequence. The reciever can determine
      which bit was the original by averaging the recieved data using the following
      formula:
      \begin{equation*}
      \ddfrac{\sum_{m=1}^{M}{Z_{i,m}^*\cdot c_m}}{M}
      \end{equation*}
      Where $M$ is the length of the chipping sequence,
      $Z_{i}$ is the current slot inside the stream and $c$ is the
      chipping sequence. Note that $M$-bits inside the encoded stream will
      decode into a single bit.
\end{enumerate}

\section*{Question 6}
Topic: Active Networking

Great job outlining this topic. I wonder if hardware currently found in critical points of the internet could handle the increased processing required to implement active networking. Obviously, per-packet, the processing is quite basic, however, some routers need to handle an extremely high volume of packets so interpreting software instructions could prove to bog-down high traffic nodes.


\end{document}
